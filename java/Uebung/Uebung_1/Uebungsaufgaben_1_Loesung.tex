\documentclass[12pt,a4paper,ngerman]{scrartcl}
\usepackage[ngerman]{babel}
 \usepackage{amsfonts,amssymb,amsmath}
\pagestyle{plain}

\subject{\vspace{-3cm}The GAME}
\title{Übungsaufgaben 1 - Lösung}
\subtitle{Zahlensysteme}
\date{}

\begin{document}
	\maketitle
	
	\paragraph{Aufgabe 1}\mbox{}\\
	Bekanntlich heißen Zahlensysteme ja auch Stellenwertsysteme. Begründe warum.
	
	\subparagraph{Lösung:} \emph{Die Stellen, an denen Ziffern stehen haben eine Stellenwertigkeit.}\\ Für interessierte Lesende, hier mehr Infos: Eine Zahl sieht im allgemeinen wie folgt aus (Nennt sich $b$-adische Form): $a_B := z_nz_{n-1}z_{n-2} \dots z_0$ mit $z \in \Sigma$ und $n \in \mathbb{N}_0$. $\Sigma$ ist hierbei als ein Alphabet zu betrachten, aus dem die Zahl also zusammen gesetzt werden kann. Für das Binärsystem wäre $\Sigma = \{0,1\}$, für das Dezimalsystem $\Sigma = \{0,1,2,3,4,5,6,7,8,9\}$. Die Basis eines Zahlensystems ist außerdem nichts anderes als die Anzahl der Elemente im Alphabet $\Sigma$, also $B = |\Sigma|$. Die Stellenwertigkeit ergibt sich nun aus der Formel $B^n$, wobei $n$ der Stellenindex ist (siehe oben $a$). Die Stelle mit der Ziffer $z_5$ Stelle hat also die Stellenwertigkeit $B^5$. Mit diesen Erkenntnissen lässt sich nun auch die Zahl $a$ ausführlich mit den Stellenwertigkeiten darstellen: \\
	$a = z_n \cdot B^n + z_{n-1} \cdot B^{n-1} + z_{n-2} \cdot B^{n-2} + \dots + z_0 \cdot B^0
	= \sum_{k=0}^{n} z_k \cdot B^k$.\\
	An dieser Stelle möchte ich noch drei Dinge anmerken:
	\begin{itemize}
		\item[1)] Wir rechnen eigentlich immer im Dezimalsystem, weil uns das einfach sehr gut liegt. Aber nehmen wir mal das Beispiel $A_{H}$. Ausführlich wäre das ja $A_{H} = A \cdot 16^0$, da $A$ ja die Ziffer an der Stelle mit der Stellenwertigkeit $16^0$ ist. Aber wie rechnen wir bitte $A \cdot 16^0$? Nun, ich habe bisher verschwiegen, dass wir eigentlich nicht die Ziffer selbst multiplizieren (wie die Formel ja sagt), sondern eigentlich den Ziffernwert. Der Ziffernwert ist einfach eine natürliche Zahl und das Bild der Abbildung $f(z_n) = \begin{cases}
			\Sigma \rightarrow \mathbb{N}_0\\
			z_n \rightarrow n
		\end{cases}$.\\
		Beispiel mit $\Sigma = \{0_0,\dots,9_9,A_{10},B_{11},\dots,F_{15}\}: f(A_{10}) = 10$, also jeder Ziffer wird die Position zugeordnet und die Position in der Menge ist also der Ziffernwert (Achtung: $\Sigma$ muss eine geordnete Menge sein: $(\Sigma, \le)$). Unser Beispiel $A_H$ wäre also eigentlich $A_{H} = f(A) \cdot 16^0 = 10 \cdot 16^0$. Und das können wir wieder gut rechnen, da das im Dezimalsystem ist.
		
		\item[2)] Genau genommen sind Zahlensysteme und Stellenwertsysteme nicht das gleiche, sondern ein Stellenwertsystem ist ein positionelles Zahlensystem. Es gibt noch andere Arten von Zahlensystemen.
		
		\item[3)] Ich habe oben den Begriff Alphabet verwendet. Tatsächlich sind wir hier in der theoretischen Informatik, genauer den formalen Sprachen. Die Zahl $a$ ist nichts anderes als ein Wort, das mit Elementen aus dem Alphabet $\Sigma$ gebildet wurde. Die Menge aller möglichen Kombinationen mit den Elementen eines Alphabets $\Sigma$ nennt sich Kleenesche Hülle und wird mit $\Sigma^*$ bezeichnet. Also die Kleenesche Hülle von $\Sigma = \{0,1\}$ ist $\Sigma^* = \{0,1,10,11,100,101,\dots\}$ und das sieht doch irgendwie nach der Menge aller Binärzahlen aus, oder? Die Zahl $a \in \Sigma^*$ kann also auch als Wort einer formalen Sprache aufgefasst werden. Hier ist die formale Sprache sogar eine sogenannte reguläre Sprache. Aber genug theoretische Informatik.
	\end{itemize}
	
	\paragraph{Aufgabe 2}\mbox{}\\
	Das Dezimalsystem ist da System, mit dem wir hauptsächlich im Alltag umgehen und auch gerechnet wird. Schreibe folgende Zahlen im Dezimalsystem ausführlich, also als Summe, hin:
	\begin{itemize}
		\item[a)] $5_{10}$
		\item[b)] $46_{10}$
		\item[c)] $198_{10}$
		\item[d)] $10359_{10}$
	\end{itemize}

	\subparagraph{Lösung:} 
	\begin{itemize}
		\item[a)] $5_{10} = 5 \cdot 10^0$
		\item[b)] $46_{10} = 4 \cdot 10^1 + 6 \cdot 10^0$
		\item[c)] $198_{10} = 1 \cdot 10^2 + 9 \cdot 10^1 + 8 \cdot 10^0$
		\item[d)] $10359_{10} = 1 \cdot 10^4 + 0 \cdot 10^3 + 3 \cdot 10^2 + 5 \cdot 10^1 + 9 \cdot 10^0$
	\end{itemize}
		
	\paragraph{Aufgabe 3}\mbox{}\\	
		Das Binärsystem ist in der Informatik, Informationstechnik und andere eng verwandte Wissenschaften und Technologiebereichen weit verbreitet, da moderne Rechensysteme (z.B. Computer) im Grunde ausschließlich damit arbeiten. Daher kann es öfter mal vorkommen, dass Binär- in Dezimalzahlen umgerechnet werden müssen. Rechne um: 
	\begin{itemize}
		\item[a)] $10_{2}$
		\item[b)] $110_{2}$
		\item[c)] $11001_{2}$
		\item[d)] $100011110_{2}$
	\end{itemize}
	
	\subparagraph{Lösung:} 
	\begin{itemize}
		\item[a)] $10_{2} = 1 \cdot 2^1 = 2_{10}$
		\item[b)] $110_{2} = 1 \cdot 2^2 + 1 \cdot 2^1 = 6_{10}$
		\item[c)] $11001_{2} = 1 \cdot 2^4 + 1 \cdot 2^3 + 1 \cdot 2^0 = 25_{10}$
		\item[d)] $100011110_{2} 
		           = 1 \cdot 2^8 + 1 \cdot 2^4 + 1 \cdot 2^3 + 1 \cdot 2^2 +  1 \cdot 2^1
		           = 286_{10}$
	\end{itemize}

	\paragraph{Aufgabe 4}\mbox{}\\	
	Auch die Rückrichtung ist relevant! Rechne die Dezimalzahlen aus 2) ins Binärsystem um.
	
	\subparagraph{Lösung:} 
	\begin{itemize}
		\item[a)] $5_{10} = 101_2$
	  	\begin{alignat*}{3}
		  	5 / 2 &= 2 && \quad R\ 1\\
		  	2 / 2 &= 1 && \quad R\ 0\\
		  	1 / 2 &= 0 && \quad R\ 1
	  	\end{alignat*}
	  	
		\item[b)] $46_{10} = 10\ 1110_2$
		\begin{alignat*}{3}
			46 / 2 &= 23 && \quad R\ 0\\
			23 / 2 &= 11 && \quad R\ 1\\
			11 / 2 &= 5 && \quad R\ 1\\
			5 / 2 &= 2 && \quad R\ 1\\
			2 / 2 &= 1 && \quad R\ 0\\
			1 / 2 &= 0 && \quad R\ 1
		\end{alignat*}
		
		
		\item[c)] $198_{10} = 1100\ 0110_2$
		\begin{alignat*}{3}
			198 / 2 &= 99 && \quad R\ 0\\
			99 / 2 &= 49 && \quad R\ 1\\
			49 / 2 &= 24 && \quad R\ 1\\
			24 / 2 &= 12 && \quad R\ 0\\
			12 / 2 &= 6 && \quad R\ 0\\
			6 / 2 &= 3 && \quad R\ 0\\
			3 / 2 &= 1 && \quad R\ 1\\
			1 / 2 &= 0 && \quad R\ 1
		\end{alignat*}
		
		
		\item[d)] $10359_{10} = 10\ 1000\ 0111\ 0111_2$
		\begin{alignat*}{3}
			10359 / 2 &= 5179 && \quad R\ 1\\
			5179 / 2 &= 2589 && \quad R\ 1\\
			2589 / 2 &= 1294 && \quad R\ 1\\
			1294 / 2 &= 647 && \quad R\ 0\\
			647 / 2 &= 323 && \quad R\ 1\\
			323 / 2 &= 161 && \quad R\ 1\\
			161 / 2 &= 80 && \quad R\ 1\\
			80 / 2 &= 40 && \quad R\ 0\\
			40 / 2 &= 20 && \quad R\ 0\\
			20 / 2 &= 10 && \quad R\ 0\\
			10 / 2 &= 5 && \quad R\ 0\\
			5 / 2 &= 2 && \quad R\ 1\\
			2 / 2 &= 1 && \quad R\ 0\\
			1 / 2 &= 0 && \quad R\ 1
		\end{alignat*}
	\end{itemize}
	
	\paragraph{Aufgabe 5}\mbox{}\\
	Das Oktalsystem kommt zwar seltener vor, aber es ist eine gute Übung. Rechne folgende Oktalzahlen in Dezimalzahlen um:
	\begin{itemize}
		\item[a)] $5_{8}$
		\item[b)] $46_{8}$
		\item[c)] $175_{8}$
		\item[d)] $10354_{8}$
	\end{itemize}

	\subparagraph{Lösung:} 
	\begin{itemize}
		\item[a)] $5_{8} = 5 \cdot 8^0 = 5_{10}$
		\item[b)] $46_{8} = 4 \cdot 8^1 + 6 \cdot 8^0 = 38_{10}$
		\item[c)] $175_{8} = 1 \cdot 8^2 + 7 \cdot 8^1 + 5 \cdot 8^0 = 125_{10}$
		\item[d)] $10354_{8} = 1 \cdot 8^4 + 3 \cdot 8^2 + 5 \cdot 8^1 + 4 \cdot 8^0 = 4332_{10}$
	\end{itemize}

	\paragraph{Aufgabe 6}\mbox{}\\
	Rechne die Dezimalzahlen aus 2) in Oktalzahlen um.
	
	\subparagraph{Lösung:} 
	\begin{itemize}
		\item[a)] $5_{10} = 5_8$
		\begin{alignat*}{3}
		5 / 8 &= 0 && \quad R\ 5
		\end{alignat*}
		
		\item[b)] $46_{10} = 56_8$
		\begin{alignat*}{3}
		46 / 8 &= 5 && \quad R\ 6\\
		5 / 8 &= 0 && \quad R\ 5
		\end{alignat*}
		
		
		\item[c)] $198_{10} = 306_8$
		\begin{alignat*}{3}
		198 / 8 &= 24 && \quad R\ 6\\
		24 / 8 &= 3 && \quad R\ 0\\
		3 / 8 &= 0 && \quad R\ 3
		\end{alignat*}
		
		
		\item[d)] $10359_{10} = 24167_8$
		\begin{alignat*}{3}
		10359 / 8 &= 1294 && \quad R\ 7\\
		1294 / 8 &= 161 && \quad R\ 6\\
		161 / 8 &= 20 && \quad R\ 1\\
		20 / 8 &= 2 && \quad R\ 4\\
		2 / 8 &= 0 && \quad R\ 2
		\end{alignat*}
	\end{itemize}
	
	\paragraph{Aufgabe 7}\mbox{}\\
	Das hexadezimale Zahlensystem ist neben dem Binärsystem enorm wichtig in Informatik und Co. Daher rechne bitte folgende Hexadezimalzahlen ins Dezimalsystem um:
	\begin{itemize}
		\item[a)] $5_{16}$
		\item[b)] $FF_{16}$
		\item[c)] $A6E_{16}$
		\item[d)] $7B9F3C_{16}$
	\end{itemize}

	\subparagraph{Lösung:} 
	\begin{itemize}
		\item[a)] $5_{16} = 5 \cdot 16^0 = 5_{10}$
		\item[b)] $FF_{16} = 15 \cdot 16^1 + 15 \cdot 16^0 = 255_{10}$
		\item[c)] $A6E_{16} = 10 \cdot 16^2 + 6 \cdot 16^1 + 14 \cdot 16^0 = 2670_{10}$
		\item[d)] $7B9F3C_{16} = 7 \cdot 16^5 + 11 \cdot 16^4 + 9 \cdot 16^3 + 15 \cdot 16^2 + 3 \cdot 16^1 + 12 \cdot 16^0 = 8101692_{10} $
	\end{itemize}

	\paragraph{Aufgabe 8}\mbox{}\\
	Ihr kennt das Spiel: Die Dezimalzahlen aus 2) in Hexadezimalzahlen umrechnen.
	
	\subparagraph{Lösung:} 
	\begin{itemize}
		\item[a)] $5_{10} = 5_{16}$
		\begin{alignat*}{3}
		5 / 16 &= 0 && \quad R\ 5
		\end{alignat*}
		
		\item[b)] $46_{10} = 2E_{16}$
		\begin{alignat*}{3}
		46 / 16 &= 2 && \quad R\ 14\\
		2 / 16 &= 0 && \quad R\ 2
		\end{alignat*}
		
		
		\item[c)] $198_{10} = C6_{16}$
		\begin{alignat*}{3}
		198 / 16 &= 12 && \quad R\ 6\\
		12 / 16 &= 0 && \quad R\ 12
		\end{alignat*}
		
		
		\item[d)] $10359_{10} = 2877_{16}$
		\begin{alignat*}{3}
		10359 / 16 &= 647 && \quad R\ 7\\
		647 / 16 &= 40 && \quad R\ 7\\
		40 / 16 &= 2 && \quad R\ 8\\
		2 / 16 &= 0 && \quad R\ 2
		\end{alignat*}
	\end{itemize}
	
	\paragraph{Aufgabe 9}\mbox{}\\
	Das schöne zwischen Binär- und Hexadezimalzahlen ist, dass man sie ganz leicht ineinander umrechnen kann. Vier Binärstellen entsprechen einer Hexadezimalstelle. Ähnlich ist das bei Binär zu Oktal auch: Drei Binärziffern entsprechen einer Oktalziffer. Im Folgenden rechne also bitte die angegebene Binärzahl sowohl in Hexadezimal, als auch Oktalzahl um (Zur Vereinfachung, habe ich bereits die Hexadezimalzahl zu je Vierer- bzw. Dreiergruppen hingeschrieben):
	\begin{itemize}
		\item[] $1110\ 1001\ 0011\ 1100\ 1101\ 1111\ 0011\ 0011\ 0011_{2}$
		\item[] $111\ 010\ 010\ 011\ 110\ 011\ 011\ 111\ 001\ 100\ 110\ 011_{2}$
	\end{itemize}

	\subparagraph{Lösung:} 
	\begin{itemize}
		\item[] $\underbrace{1110}_{E}\ 
		\underbrace{1001}_{9}\ 
		\underbrace{0011}_{3}\ 
		\underbrace{1100}_{C}\ 
		\underbrace{1101}_{D}\ 
		\underbrace{1111}_{F}\ 
		\underbrace{0011}_{3}\ 
		\underbrace{0011}_{3}\ 
		\underbrace{0011_{2}}_{3_{16}}$
		\item[] $\underbrace{111}_{7}\ 
		\underbrace{010}_{2}\ 
		\underbrace{010}_{2}\ 
		\underbrace{011}_{3}\ 
		\underbrace{110}_{6}\ 
		\underbrace{011}_{3}\ 
		\underbrace{011}_{3}\ 
		\underbrace{111}_{7}\ 
		\underbrace{001}_{1}\ 
		\underbrace{100}_{4}\ 
		\underbrace{110}_{6}\ 
		\underbrace{011_{2}}_{3_8}$
	\end{itemize}

\end{document}