\documentclass[12pt,a4paper,ngerman]{scrartcl}
\usepackage[ngerman]{babel}
\usepackage{amsfonts,amssymb,amsmath}
\pagestyle{plain}

\subject{\vspace{-3cm}The GAME}
\title{Übungsaufgaben 1}
\subtitle{Zahlensysteme}
\date{}

\begin{document}
	\maketitle
	
	\paragraph{Aufgabe 1}\mbox{}\\
	Bekanntlich heißen Zahlensysteme ja auch Stellenwertsysteme. Begründe warum.
	
	\paragraph{Aufgabe 2}\mbox{}\\
	Das Dezimalsystem ist das System, mit dem wir hauptsächlich im Alltag umgehen und in dem auch gerechnet wird. Schreibe folgende Zahlen im Dezimalsystem ausführlich, also als Summe, hin:
	\begin{itemize}
		\item[a)] $5_{10}$
		\item[b)] $46_{10}$
		\item[c)] $198_{10}$
		\item[d)] $10359_{10}$
	\end{itemize}
	
	\paragraph{Aufgabe 3}\mbox{}\\	
	Das Binärsystem ist in der Informatik, Informationstechnik und andere eng verwandte Wissenschaften und Technologiebereichen weit verbreitet, da moderne Rechensysteme (z.B. Computer) im Grunde ausschließlich damit arbeiten. Daher kann es öfter mal vorkommen, dass Binär- in Dezimalzahlen umgerechnet werden müssen. Rechne um: 
	\begin{itemize}
		\item[a)] $10_{2}$
		\item[b)] $110_{2}$
		\item[c)] $11001_{2}$
		\item[d)] $100011110_{2}$
	\end{itemize}
	
	\paragraph{Aufgabe 4}\mbox{}\\	
	Auch die Rückrichtung ist relevant! Rechne die Dezimalzahlen aus 2) ins Binärsystem um.
	
	\paragraph{Aufgabe 5}\mbox{}\\
	Das Oktalsystem kommt zwar seltener vor, aber es ist eine gute Übung. Rechne folgende Oktalzahlen in Dezimalzahlen um:
	\begin{itemize}
		\item[a)] $5_{8}$
		\item[b)] $46_{8}$
		\item[c)] $175_{8}$
		\item[d)] $10354_{8}$
	\end{itemize}

	\paragraph{Aufgabe 6}\mbox{}\\
	Rechne die Dezimalzahlen aus 2) in Oktalzahlen um.
	
	\paragraph{Aufgabe 7}\mbox{}\\
	Das hexadezimale Zahlensystem ist neben dem Binärsystem enorm wichtig in Informatik und Co. Daher rechne bitte folgende Hexadezimalzahlen ins Dezimalsystem um:
	\begin{itemize}
		\item[a)] $5_{16}$
		\item[b)] $FF_{16}$
		\item[c)] $A6E_{16}$
		\item[d)] $7B9F3C_{16}$
	\end{itemize}
	
	\paragraph{Aufgabe 8}\mbox{}\\
	Ihr kennt das Spiel: Die Dezimalzahlen aus 2) in Hexadezimalzahlen umrechnen.
	
	\paragraph{Aufgabe 9}\mbox{}\\
	Das schöne zwischen Binär- und Hexadezimalzahlen ist, dass man sie ganz leicht ineinander umrechnen kann. Vier Binärstellen entsprechen einer Hexadezimalstelle. Ähnlich ist das bei Binär zu Oktal auch: Drei Binärziffern entsprechen einer Oktalziffer. Im Folgenden rechne also bitte die angegebene Binärzahl sowohl in Hexadezimal, als auch Oktalzahl um (Zur Vereinfachung, habe ich bereits die Hexadezimalzahl zu je Vierer- bzw. Dreiergruppen hingeschrieben):
	\begin{itemize}
		\item[] $1110\ 1001\ 0011\ 1100\ 1101\ 1111\ 0011\ 0011\ 0011_{2}$
		\item[] $111\ 010\ 010\ 011\ 110\ 011\ 011\ 111\ 001\ 100\ 110\ 011_{2}$
	\end{itemize}
\end{document}