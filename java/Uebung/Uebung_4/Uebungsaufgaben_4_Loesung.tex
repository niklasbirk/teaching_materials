\documentclass[12pt,a4paper,ngerman]{scrartcl}
\usepackage[ngerman]{babel}
\usepackage{amsfonts,amssymb,amsmath,listings,color}
\pagestyle{plain}

\subject{\vspace{-3cm}The GAME}
\title{Übungsaufgaben 4}
\subtitle{Arrays - Lösung}
\date{}

\definecolor{pblue}{rgb}{0.13,0.13,1}
\definecolor{pgreen}{rgb}{0,0.5,0}
\definecolor{pred}{rgb}{0.9,0,0}
\definecolor{pgrey}{rgb}{0.46,0.45,0.48}

\lstset{language=Java,
	showspaces=false,
	showtabs=false,
	breaklines=true,
	showstringspaces=false,
	breakatwhitespace=true,
	commentstyle=\color{pgreen},
	keywordstyle=\color{pblue},
	stringstyle=\color{pred},
	basicstyle=\ttfamily,
	tabsize=4,
	moredelim=[il][\textcolor{pgrey}]{$$},
	moredelim=[is][\textcolor{pgrey}]{\%\%}{\%\%}
}

\begin{document}
	\maketitle
	
	\paragraph{Aufgabe 1}\mbox{}\\
	Gegeben sind verschiedene Längen bzw. Größen von Arrays. Gebe die jeweiligen Start- und Endindizes an.
	
	\begin{itemize}
		\item[a)] 1
		\item[b)] 42
		\item[c)] 13
		\item[d)] 101
		\item[e)] 3141592
		\item[f)] 0 $\qquad$ (Achtung fies!)
	\end{itemize}
	
	\subparagraph{Lösung:}
	\begin{itemize}
		\item[a)] Startindex: 0; Endindex: 0
		\item[b)] Startindex: 0; Endindex: 41
		\item[c)] Startindex: 0; Endindex: 12
		\item[d)] Startindex: 0; Endindex: 100
		\item[e)] Startindex: 0; Endindex: 3141591
		\item[f)] Hat keinen Start- und Endindex
	\end{itemize}
	
	\paragraph{Aufgabe 2}\mbox{}\\
	Gegeben sei ein Array der Länge 100. Welchen Index haben die folgenden "Plätze".
	
	\begin{itemize}
		\item[a)] 1
		\item[b)] 42
		\item[c)] 13
		\item[d)] 101
	\end{itemize}
	
	\subparagraph{Lösung:}
	\begin{itemize}
		\item[a)] Index: 0
		\item[b)] Index: 41
		\item[c)] Index: 12
		\item[d)] Außerhalb des Arrays
	\end{itemize}
	
	\paragraph{Aufgabe 3}\mbox{}\\
	Gegeben sei ein Array der Länge $n$. Ganz allgemein, welchen Index hat die Stelle $k$, wobei $k \le n$?
	
	\subparagraph{Lösung:}
	$k-1$
	
	\paragraph{Aufgabe 4}\mbox{}\\
	Erstelle für folgende Angaben die Arrays in Java-Code.
	
	\begin{itemize}
		\item[a)] Datentyp: \emph{int}; Länge: \emph{18}
		\item[b)] Datentyp: \emph{String}; Länge: \emph{23}
		\item[c)] Kommazahlen, Länge: 5001
		\item[d)] Datentyp: \emph{char}; Länge: \emph{7}
	\end{itemize}
	
	\subparagraph{Lösung:}
	\begin{itemize}
		\item[a)] \begin{lstlisting}
		int[] arr = new int[18];
		\end{lstlisting}
		
		\item[b)] \begin{lstlisting}
		String[] arr = new String[23];
		\end{lstlisting}
		
		\item[c)] \begin{lstlisting}
		float[] arr = new float[5001];
		double[] arr = new double[5001];
		\end{lstlisting}
		
		\item[d)] \begin{lstlisting}
		char[] arr = new char[7];
		\end{lstlisting}
	\end{itemize}
	
	\paragraph{Aufgabe 5}\mbox{}\\
	Erstelle folgende Arrays in Java, die als Inhalt die dastehenden Werte haben. Wir haben zwei Varianten kennen gelernt. Versuche beide zu verwenden.
	
	\begin{itemize}
		\item[a)] Datentyp: \emph{int}; Werte: \emph{0, 9, 2, 8, 3, 7, 4, 6, 5}
		\item[b)] Datentyp: \emph{long}; Werte: \emph{-100, 100, -50, 50, 123456789}
		\item[c)] Datentyp: \emph{String}; Werte: \emph{''Ich'', ''liebe'', ''Josh''}
		\item[d)] Datentyp: \emph{double}; Werte: \emph{3.141, 0.0005926, 5358.9793, Math.pi}
	\end{itemize}
	
	\subparagraph{Lösung:}
	\begin{itemize}
		\item[a)] \begin{lstlisting}
		int[] arr = new int[] {
			0, 9, 2, 8, 3, 7, 4, 6, 5
		};
		\end{lstlisting}
		
		\item[b)] \begin{lstlisting}
		long[] arr = new long[] {
			-100, 100, -50, 50, 123456789
		};
		\end{lstlisting}
		
		\item[c)] \begin{lstlisting}
		String arr = new String[3];
		arr[0] = "Ich";
		arr[1] = "liebe";
		arr[2] = "Josh";
		\end{lstlisting}
		
		\item[d)] \begin{lstlisting}
		double[] arr = new double[] {
			3.141, 0.0005926, 5358.9793, Math.pi
		};
		\end{lstlisting}
	\end{itemize}
	
	\paragraph{Aufgabe 6}\mbox{}\\
	Erstelle ein Programm, das das Array \emph{\{0, 1, 2, 3, 4, 5, 6, 7, 8, 9\}} in umgekehrter Reihenfolge ausgibt.
	
	\subparagraph{Lösung:}\mbox{}
	\begin{lstlisting}
	int[] arr = int[] {
		0, 1, 2, 3, 4, 5, 6, 7, 8, 9
	};
	
	for (int i = arr.length - 1; i >= 0; i--)
	{
		System.out.println(arr[i]);
	}
	\end{lstlisting}
	
	\newpage
	
	\paragraph{Aufgabe 7}\mbox{}\\
	Schreibe ein Programm, das von der Kommandozeile zwei Zahlen einliest. Diese Zahlen sollen als Start- und Endwert für Zufallszahlen dienen. In ein Array der Länge 10 lässt du dann 10 Zufallszahlen, die zwischen dem Start- und Endwert liegen, reinschreiben und gibst das Array aus.
	Für die Zufallszahlen kannst du folgendes verwenden:
	
	\begin{lstlisting}
	Random rand = new Random();
	
	nextInt(int bound)
	// Returns a pseudorandom, uniformly distributed int value between 0 (inclusive) and the specified value (exclusive), drawn from this random number generator's sequence.
	\end{lstlisting}
	
	\subparagraph{Lösung:}\mbox{}
	\begin{lstlisting}
	int rand = new Random();
	int scan = new Scanner(System.in);
	int arr = new int[10];
	
	System.out.println("Startwert: ");
	int start = scan.nextInt();
	System.out.println("Endwert: ");
	int end = scan.nextInt();
	
	for (int i = 0; i < arr.length; i++)
	{
		arr[i] = rand.nextInt(end - start + 1) + start;
	}
	
	for (int i = 0; i < arr.length; i++)
	{
		System.out.println(value);
	}
	\end{lstlisting}
	
	\paragraph{Aufgabe 8}\mbox{}\\
	Du hast ein Array mit der Länge 5 und mit folgendem Inhalt: \emph{\{0, 1, 2, 3, 4\}}. Du möchtest nun zusätzlich die Zahlen 7 und 8 abspeichern. Welche Möglichkeiten gibt es, das zu tun?
	
	\subparagraph{Lösung:}
	Man kann ein neues größeres Array anlegen und die Zahlen des alten Arrays kopieren und die neuen Zahlen hinzufügen.
	
	\paragraph{Aufgabe 9}\mbox{}\\
	Schreibe deine Ideen auf, für was man 2D-Arrays verwenden könnte.
	
	\subparagraph{Lösung:} 
	\begin{itemize}
		\item[-] Spielbrett (Schachbrett, TicTacToe, etc.)
		\item[-] Andere Arten von Tabellen
		\item[-] etc.
	\end{itemize}

\end{document}
