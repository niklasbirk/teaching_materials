\documentclass[12pt,a4paper,ngerman]{scrartcl}
\usepackage[ngerman]{babel}
\usepackage{amsfonts,amssymb,amsmath}
\usepackage{hyperref}
\pagestyle{plain}

\subject{\vspace{-3cm}The GAME}
\title{Übungsaufgaben 6}
\subtitle{OOP und Klassen}
\date{}

\begin{document}
	\maketitle
	
	\paragraph{Aufgabe 1}\mbox{}
	\label{A1}
	Modelliere das Objekt ''Katze'' als Java-Code, also schreibe eine Klasse, mithilfe der nachfolgenden Beschreibung.
	\begin{itemize}
		\item[] Eine Katze \emph{hat} eine Anzahl an Pfoten, Augen, Ohren und ein Gewicht. 	
				Katzen haben von Geburt an Fellfarben.
				Außerdem gibt es entweder den Zustand \emph{tod} oder \emph{lebendig}.
		
		\item[]	Die Katze \emph{kann} laufen und fressen.\\
				Beim Fressen kann ihr eine bestimmte Menge an Futter übergeben werden, wodurch sie schwerer wird.
	\end{itemize}
	
	\paragraph{Aufgabe 2}\mbox{}
	Erweitere die Katze aus \hyperref[A1]{Aufgabe 1}:\\
	Eine Katze verliert Gewicht, je nachdem wie weit sie gelaufen ist.
	Falls die Katze schwerer als 10 kg ist, stirbt sie an den Folgen des Übergewichts (Beachte: Eine tote Katze kann nicht mehr fressen, ...).
	
	\paragraph{Aufgabe 3}\mbox{}
	Modelliere das Objekt ''Person'' als Java Klasse vollständig, d.h. versuche dabei alle möglichen Member und Methoden zu finden.
	
	\paragraph{Aufgabe 4}\mbox{}
	Modelliere die Member und Methoden des Objekts ''Bankautomat'' als Text. Als Hilfe kannst du die Begriffe \emph{hat} und \emph{kann} verwenden. Danach programmierst du das.

\end{document}
