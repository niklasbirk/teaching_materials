\documentclass[12pt,a4paper,ngerman]{scrartcl}
\usepackage[ngerman]{babel}
\usepackage{amsfonts,amssymb,amsmath,listings,color}
\pagestyle{plain}

\subject{\vspace{-3cm}The GAME}
\title{Übungsaufgaben 3}
\subtitle{Komplemente und Kontrollstrukturen - Lösung}
\date{}

\definecolor{pblue}{rgb}{0.13,0.13,1}
\definecolor{pgreen}{rgb}{0,0.5,0}
\definecolor{pred}{rgb}{0.9,0,0}
\definecolor{pgrey}{rgb}{0.46,0.45,0.48}

\lstset{language=Java,
	showspaces=false,
	showtabs=false,
	breaklines=true,
	showstringspaces=false,
	breakatwhitespace=true,
	commentstyle=\color{pgreen},
	keywordstyle=\color{pblue},
	stringstyle=\color{pred},
	basicstyle=\ttfamily,
	tabsize=4,
	moredelim=[il][\textcolor{pgrey}]{$$},
	moredelim=[is][\textcolor{pgrey}]{\%\%}{\%\%}
}

\begin{document}
	\maketitle
	
	\paragraph{Aufgabe 1}\mbox{}\\
	Führe für folgende Binärzahlen die bitweise Negation durch:
	\begin{itemize}
		\item[a)] $0_2$
		\item[b)] $0000_2$
		\item[c)] $1111_2$
		\item[d)] $1010_2$
		\item[e)] $1100\ 0011\ 1001\ 1011\ 0111_2$
	\end{itemize}

	\subparagraph{Lösung:} 
	\begin{itemize}
		\item[a)] $\sim 0_2 = 1_2$
		\item[b)] $\sim 0000_2 = 1111_2$
		\item[c)] $\sim 1111_2 = 0000_2$
		\item[d)] $\sim 1010_2 = 0101_2$
		\item[e)] $\sim 1100\ 0011\ 1001\ 1011\ 0111_2 = 0011\ 1100\ 0110\ 0100\ 1000_2$
	\end{itemize}
	
	\paragraph{Aufgabe 2}\mbox{}\\
	Wie lautet das Einskomplement folgender Zahlen bei 8 Bit (Man gebe zudem die Dezimalzahlen der unten stehenden Binärzahlen und ihres Einskomplements an):
	\begin{itemize}
		\item[a)] $0000\ 0101_2$
		\item[b)] $0000\ 1111_2$
		\item[c)] $0010\ 0000_2$
		\item[d)] $0111\ 1111_2$
	\end{itemize}

	\subparagraph{Lösung:} 
	\begin{itemize}
		\item[a)] $\sim 5_{10} = \sim 0000\ 0101_2 = 1111\ 1010_2 = -5_{10}$
		\item[b)] $\sim 15_{10} = \sim 0000\ 1111_2 = 1111\ 0000_2 = -15_{10}$
		\item[c)] $\sim 32_{10} = \sim 0010\ 0000_2 = 1101\ 1111_2 = -32_{10}$
		\item[d)] $\sim 127_{10} = \sim 0111\ 1111_2 = 1000\ 0000_2 = -127_{10}$
	\end{itemize}
	
	\paragraph{Aufgabe 3}\mbox{}\\	
	Man gebe die Dezimalzahl zu den unten stehenden Einskomplementen an (8 Bit):
	\begin{itemize}
		\item[a)] $1000\ 0101_2$
		\item[b)] $1000\ 1111_2$
		\item[c)] $1010\ 0000_2$
		\item[d)] $1111\ 1111_2$
	\end{itemize}

	\subparagraph{Lösung:} 
	\begin{itemize}
		\item[a)] $1000\ 0101_2 = -122_{10}$
		\item[b)] $1000\ 1111_2 = -112_{10}$
		\item[c)] $1010\ 0000_2 = -95_{10}$
		\item[d)] $1111\ 1111_2 = -0_{10}$
	\end{itemize}
	
	\paragraph{Aufgabe 4}\mbox{}\\	
	Was stört bei der Einskomplement-Darstellung?
	
	\subparagraph{Lösung:} Für die $0$ gibt es eine doppelte Darstellung, nämlich $00 \dots 00_2 = +0_{10}$ und $11 \dots 11_2 = -0_{10}$. Außerdem ist der Wertebereich um eins verkleinert.
	
	\paragraph{Aufgabe 5}\mbox{}\\
	Ermittelt zu folgenden Zahlen die Zweikomplement-Darstellung bei 8 Bit (Binärzahlen und Dezimalzahlen angeben).
	\begin{itemize}
		\item[a)] $5$
		\item[b)] $15$
		\item[c)] $32$
		\item[d)] $127$
	\end{itemize}

	\subparagraph{Lösung:} 
	\begin{itemize}
		\item[a)] $5_{10} = 0000\ 0101_2\\
			\Rightarrow \sim 0000\ 0101_2 + 1_2 = 1111\ 1010_2 + 1_2 = 1111\ 1011_2 = -5_{10}$
		\item[b)] $15_{10} = 0000\ 1111_2\\
			\Rightarrow \sim 0000\ 1111_2 + 1_2 = 1111\ 0000_2 + 1_2 = 1111\ 0001_2 = -15_{10}$
		\item[c)] $32_{10} = 0010\ 0000_2\\
			\Rightarrow \sim 0010\ 0000_2 + 1_2 = 1101\ 1111_2 + 1_2 = 1110\ 0000_2 = -32_{10}$
		\item[d)] $127_{10} = 0111\ 1111_2\\
			\Rightarrow \sim 0111\ 1111_2 + 1_2 = 1000\ 0000_2 + 1_2 = 1000\ 0001_2 = -127_{10}$
	\end{itemize}

	\paragraph{Aufgabe 6}\mbox{}\\
	Rechnet folgende Ausdrücke aus. Die Zahlen sind dabei alle in der Zweikomplement-Darstellung gegeben (8 Bit).
	\begin{itemize}
		\item[a)] $5 + (-21)$
		\item[b)] $15 + (-5)$
		\item[c)] $32 + 4$
		\item[d)] $127 + -(127)$
	\end{itemize}
	
	\subparagraph{Lösung:} 
	\begin{itemize}
		\item[a)] $5_{10} = 0000\ 0101_2,\ -21_{10} = 1110\ 1011_2\\
			\Rightarrow 0000\ 0101_2 + 1110\ 1011_2 = 1111\ 0000_2 = -16_{10}$
		\item[b)] $15_{10} = 0000\ 1111_2,\ -5_{10} = 1111\ 1011_2\\
			\Rightarrow 0000\ 1111_2 + 1111\ 1011_2 = 0000\ 1010_2 = 10_{10}$
		\item[c)] $32_{10} = 0010\ 0000_2,\ 4_{10} = 0000\ 0100_2\\
			\Rightarrow 0010\ 0000_2 + 0000\ 0100_2 = 0010\ 0100_2 = 36_{10}$
		\item[d)] $127_{10} = 0111\ 1111_2,\ -127_{10} = 1000\ 0001_2\\
			\Rightarrow 0111\ 1111_2 + 1000\ 0001_2 = 0000\ 0000_2 = 0_{10}$
	\end{itemize}
	
	\paragraph{Aufgabe 7}\mbox{}\\
	Rechnet folgende Ausdrücke aus. Die Zahlen sind dabei alle \emph{nicht} in der Zweikomplement-Darstellung gegeben, d.h. die Zahlen (die nach dem Minus) sind umzurechnen (8 Bit).
	\begin{itemize}
		\item[a)] $19 - 28$
		\item[b)] $56 - 9$
		\item[c)] $1 - 2$
		\item[d)] $3 - 127$
	\end{itemize}
	
	\subparagraph{Lösung:} 
	\begin{itemize}
		\item[a)] $19_{10} = 0001\ 0011_2,\ 28_{10} = 0001\ 1100_2\\
			\Rightarrow \sim 0001\ 1100_2 + 1_2 = 1110\ 0011_2 + 1_2 = 1110\ 0100_2 = -28_{10}\\
			\Rightarrow 19_{10} + (-28_{10}) = 0001\ 0011_2 + 1110\ 0100_2 = 1111\ 0111 = -9_{10}$
		\item[b)] $56_{10} = 0011\ 1000_2,\ 9_{10} = 0000\ 1001_2\\
			\Rightarrow \sim 0000\ 1001_2 + 1_2 = 1111\ 0110_2 + 1_2 = 1111\ 0111_2 = -9_{10}\\
			\Rightarrow 56_{10} + (-9_{10}) = 0011\ 1000_2 + 1111\ 0111_2 = 0010\ 1111_2 = 47_{10}$
		\item[c)] $1_{10} = 0000\ 0001_2,\ 2_{10} = 0000\ 0010_2\\
			\Rightarrow \sim 0000\ 0010_2 + 1_2 = 1111\ 1101_2 + 1_2 = 1111\ 1110_2 = -2_{10}\\
			\Rightarrow 1_{10} + (-2_{10}) = 0000\ 0001_2 + 1111\ 1110_2 = 1111\ 1111_2 = -1_{10}$
		\item[d)] $3_{10} = 0000\ 0011_2,\ 127_{10} = 01111\ 1111_2\\
			\Rightarrow \sim 0111\ 1111_2 + 1_2 = 1000\ 0000_2 + 1_2 = 1000\ 0001_2 = -127_{10}\\
			\Rightarrow 3_{10} + (-127_{10}) = 0000\ 0011_2 + 1000\ 0001_2 = 1000\ 0100_2 = -124_{10}$
	\end{itemize}

	\newpage
	
	\paragraph{Aufgabe 8}\mbox{}\\
	Schreibe ein einfaches Java-Programm, welches einem Monat vom Benutzer einliest (mithilfe des „java.util.Scanner“).
	Verwende dafür die Zahlen 1 bis 12 (1 = Januar, 2 = Februar, ..., 12 = Dezember). Für jeden Monat soll nun die Anzahl der Tage ausgegeben werden.
	Nutze hierzu die \emph{einfache Verzweigung}.
	
	\subparagraph{Lösung:} 
	Eine mögliche Lösung:
	\begin{lstlisting}
	import java.util.Scanner;
	public class Nr8 
	{
		public static void main(String [] args) 
		{
			Scanner scanner = new Scanner(System.in);
			System.out.println("Enter a month (Number between 1 and 12) an I tell you the number of days");
			int month = scanner.nextInt();
			
			if (month == 2) 
			{
				System.out.println("28 Tage");
			} 
			
			if ((month == 1) || (month == 3) || (month == 5) || (month == 7) || (month == 9) || (month == 11)) 
			{
				System.out.println("31 Tage");
			}
			
			if ((month == 4) || (month == 6) || (month == 8) || (month == 10) || (month == 12)) 
			{
				System.out.println("30 Tage");
			}
		}
	}
	\end{lstlisting}
	
	\newpage
	
	\paragraph{Aufgabe 9}\mbox{}\\
	Schreibe ein einfaches Java-Programm, welches einem Monat vom Benutzer einliest (mithilfe des „java.util.Scanner“).
	Verwende dafür die Zahlen 1 bis 12 (1 = Januar, 2 = Februar, ..., 12 = Dezember). Für jeden Monat soll nun die Anzahl der Tage ausgegeben werden.
	Nutze hierzu die \emph{Mehrfachverzweigung}.
	
	\subparagraph{Lösung:} 
	Eine mögliche Lösung:
	\begin{lstlisting}
	import java.util.Scanner;
	public class Nr9 
	{	
		public static void main(String [] args) 
		{
			Scanner scanner = new Scanner(System.in);
			System.out.println("Enter a month (Number between 1 and 12) an I tell you the number of days");
			int month = scanner.nextInt();
			
			switch(month)
			{
				case 1:
					System.out.println("31 Tage");
					break;
				case 2:
					System.out.println("28 Tage");
					break;
				case 3:
					System.out.println("31 Tage");
					break;
				case 4:
					System.out.println("30 Tage");
					break;
				case 5:
					System.out.println("31 Tage");
					break;
				case 6:
					System.out.println("30 Tage");
					break;
				case 7:
					System.out.println("31 Tage");
					break;
				case 8:
					System.out.println("30 Tage");
					break;
				case 9:
					System.out.println("31 Tage");
					break;
				case 10:
					System.out.println("30 Tage");
					break;
				case 11:
					System.out.println("31 Tage");
					break;
				case 12:
					System.out.println("30 Tage");
					break;
				default:
					System.out.println("Please enter a viable number between 1 and 12 (Your number: " + month + ")");
			}	
		}
	}
	\end{lstlisting}
	
	\newpage
	
	\paragraph{Aufgabe 10}\mbox{}\\
	Kann Aufgabe 8 mit 3 oder weniger If-Statements gelöst werden? Wenn ja wie? Wenn nein warum nicht?
	
	\subparagraph{Lösung:} 
	Ja.
	\begin{lstlisting}
	import java.util.Scanner;
	public class Nr10
	{
		public static void main(String [] args)
		{
			Scanner scanner = new Scanner(System.in);
			System.out.println("Enter a month (Number between 1 and 12) an I tell you the number of days");
			int month = scanner.nextInt();
			
			if(month == 2)
			{
				System.out.println("28 Tage");
				
				if((month == 1) || (month == 3) || (month == 5) || (month == 7) || (month == 9) || (month == 11))
				{
					System.out.println("31 Tage");
				}
				
				if((month == 4) || (month == 6) || (month == 8) || (month == 10) || (month == 12))
				{
					System.out.println("30 Tage");
				}
			}
		}
	}
	\end{lstlisting}
	
	\newpage
	
	\paragraph{Aufgabe 11}\mbox{}\\
	Wie oft wird die folgende Schleife ausgeführt und warum?
	\begin{lstlisting}
	int i = 10;
	do { 
		i = i - 3;
	} while (i  > 5);
	\end{lstlisting}
	
	\subparagraph{Lösung:}
	Zwei mal, da danach die Bedingung am Ende des zweiten Durchlaufes "4 > 5" lautet und somit die Schleife beendet wird. 
	
	
	\paragraph{Aufgabe 12}\mbox{}\\
	Welche Zahlen werden bei diesen Schleifen ausgegeben? Und wie oft werden diese Ausgeführt?
	\\\\
	a) \begin{lstlisting}
	for( int i=1; i <= 10; i++ ){
		System.out.println( i );
	} 
	\end{lstlisting}
	b) \begin{lstlisting}
	for( int i=1; i <= 10; i = i+2 ){
		System.out.println( i );
	}  
	\end{lstlisting}
	c) \begin{lstlisting}
	for( int i=1; i <= 10; i = i*2 ){
		System.out.println( i );
	} 
	\end{lstlisting}
	d) \begin{lstlisting}
	for( int i=1; i < 10; i = i+2 ){
		if (i >= 5) {
			System.out.println( i );
			i--;
		} else {
			System.out.println( i );
		} 
	} 
	\end{lstlisting}
	\newpage
	e) \begin{lstlisting}
	int i = 0;
	do { 
		if (i < 4) {
			System.out.println( i );
		}
		if (i > 4) {
			System.out.println( i );
		} else {
			i--;
		} 
		i = i+2;
	} while (i  < 10);  
	\end{lstlisting}
	
	\subparagraph{Lösung:}
	\begin{itemize}
		\item[a)] $\begin{array}{c|c}
				\text{Durchlauf} & \text{Ausgabe}\\
				\hline
				1 & 1\\
				2 & 2\\
				3 & 3\\
				4 & 4\\
				5 & 5\\
				6 & 6\\
				7 & 7\\
				8 & 8\\
				9 & 9\\
				10 & 10\\
			\end{array}$
		\item[b)] $\begin{array}{c|c}
				\text{Durchlauf} & \text{Ausgabe}\\
				\hline
				1 & 1\\
				2 & 3\\
				3 & 5\\
				4 & 7\\
				5 & 9
			\end{array}$
		\item[c)] $\begin{array}{c|c}
				\text{Durchlauf} & \text{Ausgabe}\\
				\hline
				1 & 1\\
				2 & 2\\
				3 & 4\\
				4 & 8
			\end{array}$
		\item[d)] $\begin{array}{c|c}
				\text{Durchlauf} & \text{Ausgabe}\\
				\hline
				1 & 1\\
				2 & 3\\
				3 & 5\\
				4 & 6\\
				5 & 7\\
				6 & 8\\
				7 & 9
			\end{array}$
		\item[e)] $\begin{array}{c|c}
				\text{Durchlauf} & \text{Ausgabe}\\
				\hline
				1 & 0\\
				2 & 1\\
				3 & 2\\
				4 & 3\\
				5 & 5\\
				6 & 7\\
				7 & 9
			\end{array}$
	\end{itemize}
\end{document}
