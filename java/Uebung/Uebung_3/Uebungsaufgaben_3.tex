\documentclass[12pt,a4paper,ngerman]{scrartcl}
\usepackage[ngerman]{babel}
\usepackage{amsfonts,amssymb,amsmath,listings,color}
\pagestyle{plain}

\subject{\vspace{-3cm}The GAME}
\title{Übungsaufgaben 3}
\subtitle{Komplemente und Kontrollstrukturen}
\date{}

\definecolor{pblue}{rgb}{0.13,0.13,1}
\definecolor{pgreen}{rgb}{0,0.5,0}
\definecolor{pred}{rgb}{0.9,0,0}
\definecolor{pgrey}{rgb}{0.46,0.45,0.48}

\lstset{language=Java,
	showspaces=false,
	showtabs=false,
	breaklines=true,
	showstringspaces=false,
	breakatwhitespace=true,
	commentstyle=\color{pgreen},
	keywordstyle=\color{pblue},
	stringstyle=\color{pred},
	basicstyle=\ttfamily,
	tabsize=4,
	moredelim=[il][\textcolor{pgrey}]{$$},
	moredelim=[is][\textcolor{pgrey}]{\%\%}{\%\%}
}

\begin{document}
	\maketitle
	
	\paragraph{Aufgabe 1}\mbox{}\\
	Führe für folgende Binärzahlen die bitweise Negation durch:
	\begin{itemize}
		\item[a)] $0_2$
		\item[b)] $0000_2$
		\item[c)] $1111_2$
		\item[d)] $1010_2$
		\item[e)] $1100\ 0011\ 1001\ 1011\ 0111_2$
	\end{itemize}
	
	\paragraph{Aufgabe 2}\mbox{}\\
	Wie lautet das Einskomplement folgender Zahlen bei 8 Bit (Man gebe zudem die Dezimalzahlen der unten stehenden Binärzahlen und ihres Einskomplements an):
	\begin{itemize}
		\item[a)] $0000\ 0101_2$
		\item[b)] $0000\ 1111_2$
		\item[c)] $0010\ 0000_2$
		\item[d)] $0111\ 1111_2$
	\end{itemize}
	
	\paragraph{Aufgabe 3}\mbox{}\\	
	Man gebe die Dezimalzahl zu den unten stehenden Einskomplementen an (8 Bit):
	\begin{itemize}
		\item[a)] $1000\ 0101_2$
		\item[b)] $1000\ 1111_2$
		\item[c)] $1010\ 0000_2$
		\item[d)] $1111\ 1111_2$
	\end{itemize}
	
	\paragraph{Aufgabe 4}\mbox{}\\	
	Was stört bei der Einskomplement-Darstellung?
	
	\paragraph{Aufgabe 5}\mbox{}\\
	Ermittelt zu folgenden Zahlen die Zweikomplement-Darstellung bei 8 Bit (Binärzahlen und Dezimalzahlen angeben).
	\begin{itemize}
		\item[a)] $5$
		\item[b)] $15$
		\item[c)] $32$
		\item[d)] $127$
	\end{itemize}

	\paragraph{Aufgabe 6}\mbox{}\\
	Rechnet folgende Ausdrücke aus. Die Zahlen sind dabei alle in der Zweikomplement-Darstellung gegeben (8 Bit).
	\begin{itemize}
		\item[a)] $5 + (-21)$
		\item[b)] $15 + (-5)$
		\item[c)] $32 + 4$
		\item[d)] $127 + (-127)$
	\end{itemize}
	
	\paragraph{Aufgabe 7}\mbox{}\\
	Rechnet folgende Ausdrücke aus. Die Zahlen sind dabei alle \emph{nicht} in der Zweikomplement-Darstellung gegeben, d.h. die Zahlen (die nach dem Minus) sind umzurechnen (8 Bit).
	\begin{itemize}
		\item[a)] $19 - 28$
		\item[b)] $56 - 9$
		\item[c)] $1 - 2$
		\item[d)] $3 - 127$
	\end{itemize}
	
	\paragraph{Aufgabe 8}\mbox{}\\
	Schreibe ein einfaches Java-Programm, welches einem Monat vom Benutzer einliest (mithilfe des „java.util.Scanner“).
	Verwende dafür die Zahlen 1 bis 12 (1 = Januar, 2 = Februar, ..., 12 = Dezember). Für jeden Monat soll nun die Anzahl der Tage ausgegeben werden.
	Nutze hierzu die \emph{einfache Verzweigung}.
	
	\paragraph{Aufgabe 9}\mbox{}\\
	Schreibe ein einfaches Java-Programm, welches einem Monat vom Benutzer einliest (mithilfe des „java.util.Scanner“).
	Verwende dafür die Zahlen 1 bis 12 (1 = Januar, 2 = Februar, ..., 12 = Dezember). Für jeden Monat soll nun die Anzahl der Tage ausgegeben werden.
	Nutze hierzu die \emph{Mehrfachverzweigung}.
	
	\paragraph{Aufgabe 10}\mbox{}\\
	Kann Aufgabe 8 mit 3 oder weniger If-Statements gelöst werden? Wenn ja wie? Wenn nein warum nicht?
	
	\paragraph{Aufgabe 11}\mbox{}\\
	Wie oft wird die folgende Schleife ausgeführt und warum?
	\begin{lstlisting}
	int i = 10;
	do { 
		i = i - 3;
	} while (i  > 5);
	\end{lstlisting}
	
	\paragraph{Aufgabe 12}\mbox{}\\
	Welche Zahlen werden bei diesen Schleifen ausgegeben? Und wie oft werden diese Ausgeführt?
	\\\\
	a) \begin{lstlisting}
	for( int i=1; i <= 10; i++ ){
		System.out.println( i );
	} 
	\end{lstlisting}
	b) \begin{lstlisting}
	for( int i=1; i <= 10; i = i+2 ){
		System.out.println( i );
	}  
	\end{lstlisting}
	c) \begin{lstlisting}
	for( int i=1; i <= 10; i = i*2 ){
		System.out.println( i );
	} 
	\end{lstlisting}
	d) \begin{lstlisting}
	for( int i=1; i < 10; i = i+2 ){
		if (i >= 5) {
			System.out.println( i );
			i--;
		} else {
			System.out.println( i );
		} 
	} 
	\end{lstlisting}
	\newpage
	e) \begin{lstlisting}
	int i = 0;
	do { 
		if (i < 4) {
			System.out.println( i );
		}
		if (i > 4) {
			System.out.println( i );
		} else {
			i--;
		} 
		i = i+2;
	} while (i  < 10);  
	\end{lstlisting}
\end{document}